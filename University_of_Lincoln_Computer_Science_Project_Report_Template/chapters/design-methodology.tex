This section of the dissertation may vary significantly in both structure and content, depending on the type of project you are undertaking. However, it is expected that all projects have the following sections:

\begin{itemize}
    \item Project management
    \item Risk analysis
\end{itemize}

The precise structure should be discussed with your supervisor, but some suggestions for additional sections are given below. The key thing to note here is that irrespective of the project type, you should \emph{justify} the choices you've made, rather than simply choosing based on expediency or familiarity.

\section{Software development projects}
\label{sd-projects}

If the primary deliverable of your project is a software product, then you should consider subsections detailing your approaches to the following:

\begin{itemize}
    \item Software development methodology (e.g., waterfall, scrum)
    \item Toolsets and machine environments (i.e., the software and hardware used)
    \item Design (e.g., UML diagrams, database schema, prototypes)
    \item Testing (i.e., the types of testing used)
\end{itemize}

This list is not exhaustive. For example, a games design project may include a game design document. However, it must be noted that if your project contains significant software development work, then most if not all of these sections should be present.

\section{Research \emph{not} involving human participants}

For some projects, the main deliverables may come in the form of experimental results. For example, a project comparing several different algorithms may require little in the way of code, but require considerable experimentation and data analysis. As such, all methodological choices made should be documented here. Examples include:

\begin{itemize}
    \item Dataset acquisition and annotation
    \item Algorithm/model design and selection
    \item Parameter tuning
    \item Performance metrics
\end{itemize}

Again, this list is not exhaustive, and you should still include relevant sections pertaining to the software artefact listed in Section \ref{sd-projects}.

\section{Research involving human participants}

For projects involving human participants, you will need to consider a hypothesis or research question that your project will answer. In addition to the sections outlined in Section \ref{sd-projects}, you may also need to provide details of:

\begin{itemize}
    \item Participant recruitment
    \item Evidence that ethical procedures have been followed
    \item Study design (including hypotheses/research question as appropriate)
    \item Statistical analysis (i.e., how you'll analyse the raw data)
\end{itemize}

If your study involves data collection by means of questionnaires, you may also wish to specify the questions here (or refer to them in an appendix).